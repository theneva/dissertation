\section{Master thesis research
proposal}\label{master-thesis-research-proposal}

\subsection{Abstract}\label{abstract}

\ldots{}

\subsection{1 Introduction}\label{introduction}

This document is a proposal for the research project I will conduct for
my Master's thesis during the spring of 2016. It begins by listing core
aims and objectives for the project in terms of what I seek to deliver.
It then suggests a research approach to achieve this goal, and finally
presents a brief project plan.

My overarching research question is how we can best compare several
strategies for automating deployment in an environment with many teams
and services, and pick or compose a strategy that suits the context.

My motivation for this project comes from the action design
research-like work I completed in the Master programme's practise period
module. During this project, I identified automated deployment of
microservices as a time-consuming and error-prone task that is often
repeated for each new service.

A brief literature search will show that much work has already been done
on both \emph{automated deployment} and \emph{microservices}
individually, such as \autocite{virmani:2015}, \autocite{stolberg},
\autocite{savchenko:2015}, and \autocite{le:2015}. However, it appears
from the literature review I conducted then that little work is
completed on automated deployment in a microservice context. This must
be shown explicitly in my thesis.

\subsection{2 Aim and objectives}\label{aim-and-objectives}

As mentioned in the previous section, deployment in a microservice
context is a task that is both difficult and time-consuming. My
\textbf{overall aim} for the research is to help simplify this process
through three \textbf{key goals}:

\begin{enumerate}
\def\labelenumi{\arabic{enumi}.}
\tightlist
\item
  Providing further insight into how implementation of automated
  microservice deployment;
\item
  Creating a framework for analysing different strategies for automated
  deployment in the future; and
\item
  Conducting an analysis of a few popular strategies for automated
  deployment.
\end{enumerate}

In order to reach these goals, my \textbf{specific objectives} are:

\begin{enumerate}
\def\labelenumi{\arabic{enumi}.}
\tightlist
\item
  Reviewing the existing literature on deployment automation (in this
  context commonly referred to as \emph{continuous integration} or
  \emph{continuous delivery}), the microservice architectural style in
  general, how they can be tied together;
\item
  Establishing relevance by learning which factors are important to the
  industry;
\item
  Developing an initial framework for analysing strategies based on the
  findings from the literature review (1) and the case study (2);
\item
  Testing, analysing, and comparing some popular deployment automation
  strategies using the framework from (3) to (a) simplify picking or
  composing a strategy and (b) validate and mature the framework.
\end{enumerate}

\subsection{3 Research approach}\label{research-approach}

\ldots{}~through a preliminary case study at FINN.no, comprising
interviews and document review;

I suggest triangulating two research strategies:

\begin{itemize}
\tightlist
\item
  A \textbf{case study} to establish which factors are important to the
  industry to supplement the literature review; and
\item
  A \textbf{design science research} \autocite{vaishnavi:2015}
  (specifically, design \& creation \autocite{oates:2006}) project to
  test various deployment strategies.
\end{itemize}

\subsubsection{3.1 Case study: What is important to
FINN.no?}\label{case-study-what-is-important-to-finn.no}

Together with data from a literature review, I will build a first
version of a framework for analysing the pros and cons of various
strategies. This framework will be the key artefact of this thesis. In
order to find which factors are important to the industry, I will
conduct a case study at FINN.no. Using interviews and a document
analysis for data generation, I will gather data on the company's own
analysis of various strategies and how they plan to implement their own.
I will interview (at least):

\begin{enumerate}
\def\labelenumi{\arabic{enumi}.}
\tightlist
\item
  Key members of the cloud migration team responsible for implementing
  containers to understand which efforts are being made;
\item
  The team lead for one of FINN's teams that has completely deviated
  from the company's standard way to do deployment;
\item
  The person(s) responsible for the containerisation in order to uncover
  \emph{why} the effort is made and \emph{which factors} determine which
  strategy is selected; and
\item
  A key member of the cloud migration team to polish and verify my
  findings.
\end{enumerate}

These interviews will provide insight into which strategies are actually
being used for automated deployment, and how they are composed. This
initial version of the framework then be used to analyse some
implementations based on the next step.

All the listed interviews are already agreed upon by FINN.no, so the
risk of not being able to gather data is relatively small. Additionally,
if it proves unfeasible to gather data required to build a framework, I
will simply need to lean more heavily toward my findings during the
literature review. Thus, even if this phase should yield few results,
the project can continue. The obvious downside of this outcome is that
the industry is not necessarily represented strongly in building the
framework, which makes it harder to establish a strong relevance.

\subsubsection{3.2 Implementation through design science
research}\label{implementation-through-design-science-research}

In this final phase of the project, I will iteratively improve the
analysis framework for analysing deployment automation strategies
derived from the literature review and case study.

In order to test strategies, I will need to build some relatively simple
microservices that can be used for deployment. I will write these
services in a few popular programming languages using frameworks
commonly used by the industry to improve the study's generalisability. I
expect to use an Infrastructure as a
Service\footnote{A service where the hardware and network are abstracted away, but one receives access directly to the operating system(s).}
such as Amazon Web Services
(AWS)\footnote{\url{https://aws.amazon.com/}} to set up a test
environment.

The first major revision of the framework will come from implementing
and testing a strategy close to the one FINN is committed to. This test
will likely raise several new concerns to include in the framework, as
the issues must be explicitly addressed. The implementation of a few
other strategies will both help improve the analysis framework and
provide data that can help practitioners select a strategy that suits
them. Practitioners can even compose their own strategies and compare
them using the framework.

I will iteratively add to the total number of strategies I test; if I am
able to implement and test a first strategy, this phase will already
have yielded useful results. Given the time schedule presented in the
next section, this should not be a high risk. However, the study's
validity and generalisability are almost highly dependent on more than
one implemented and evaluated strategy, as the key artefact from this
phase is a \emph{comparison} of various strategies.

\subsection{4 Project plan}\label{project-plan}

\begin{figure}[htbp]
\centering
\includegraphics{http://img.ctrlv.in/img/16/01/15/56998070a7525.png}
\caption{Overall date-oriented project plan}
\end{figure}

\subsection{Potential titles}\label{potential-titles}

So far, I only have one potential title. Suggestions will be kept in and
added to a live document during the writing process.

\begin{itemize}
\tightlist
\item
  A comparison of automated deployment strategies
\end{itemize}
