\chapter*{Abstract}

\textbf{Purpose}

The evolution of service-oriented computing, and the microservice architecture in particular, has introduced a new layer of complexity the already challenging task of continuously delivering changes to the end users. The rise of cloud computing has turned scalable hardware into a commodity, but also imposes some requirements on the software development process. The study presented in this thesis completes two key objectives.

First, a framework is developed as a tool for software architects to evaluate and compare strategies for Continuous Deliver of microservices in the context of their organisation and project.

Second, the framework is applied to two strategies for Continuous Delivery: fully manual deployment, and scripted automated deployment.

\textbf{Methodology}

The study follows the process of Design Science Research by developing an initial framework for evaluation and comparison of strategies for Continuous Deployment of microservices. The initial framework is based on data from the existing literature and a case study at the Norwegian company FINN.no. The framework is then iteratively improved by developing and deploying a system of microservices using manual and scripted automated deployment. 

\textbf{Findings}

A framework for evaluating and comparing approaches to continuous microservice delivery was developed and iteratively improved through application to a realistic system of microservices. This process uncovered that there is a distinct gap in the literature on Continuous Delivery in relation to service-oriented computing and microservices. The literature is mainly focused on quantifiable metrics such as number of manual steps and Lines Of Code required to make a change. The industry, on the other hand, appear to be more focused on qualitative metrics such as increasing the productivity of their developers. These are common goals, but must be measured using different approaches.

The framework proves to an extent that it is possible to efficiently evaluate and compare strategies for continuously delivering microservices. However, further testing of the framework in real business contexts, as well as additional work to bridge the gap in the academic literature are required to prove the worth of the concluded approach to evaluating and comparing strategies for Continuous Delivery in a microservice context.

\textbf{Keywords}: microservices, software architecure, deployment, Continuous Delivery, DevOps, quality assurance
